\documentclass[10pt,letterpaper]{article}
\usepackage[left=2cm,right=2cm,top=2cm,bottom=2cm]{geometry}
\usepackage[utf8]{inputenc}
\usepackage[spanish]{babel}
\usepackage{apacite}
\usepackage[numbib,nottoc]{tocbibind}
\usepackage{graphicx}
\usepackage{url}

\author{Rony J. Letona}
\title{Taller de Computaci\'on Cient\'ifica para Ciencias Qu\'imicas - Programa}

\renewcommand\bibname{REFERENCIAS}

\begin{document}

\maketitle

\section{Introducci\'on}
El taller que se propone pretende dar la oportunidad, a quienes atiendan, a incursionar en el campo de la inform\'atica qu\'imica, quimioinform\'atica y qu\'imica computacional. Se comienza con conceptos b\'asicos, pero esenciales, para la comprensi\'on y uso de un ordenador. Luego se procede a escalar hacia el uso de paquetes, librer\'ias y programas espec\'ificos. Finalmente se llega a hacer uso de programas especializados para construir modelos y realizar simulaciones con diversas aplicaciones. Esto se distribuye a lo largo de 13 sesiones en donde cada participante hace ejercicios sobre cada tema visto a modo de tener contacto directo con cada interfase y cada diferente ambiente virtual. Al terminar se espera que cada uno tenga una mejor idea de lo que se puede hacer en el campo de la computaci\'on cient\'ifica para poder hacer uso de \'el en su formaci\'on acad\'emica, as\'i como en su vida profesional.

\section{Objetivos}
\subsection{General}
Que quienes atiendan al taller ampl\'ien su visi\'on sobre el alcance y las aplicaciones del uso de tecnolog\'ias de la informaci\'on orientadas a qu\'imica y puedan aplicar estos conceptos y pr\'acticas en sus cursos o ambiente profesional.
\subsection{Espec\'ificos}
Que quienes atiendan al taller:
\begin{enumerate}
\item Aprendan sobre el sistema operativo Linux, hagan uso de \'el y comprendan sobre la conveniencia de este en el campo de las ciencias exactas.
\item Aprendan sobre el uso del shell (capa de l\'inea de comando) \emph{bash} y hagan uso de \'el para facilitar muchas tareas cotidianas.
\item Se familiaricen con las bases de datos relacionales y hagan uso del lenguaje SQL para interactuar con ellas.
\item Conozcan sobre el lenguaje \LaTeX\ y lo utilicen en la elaboraci\'on de documentos cient\'ificos, reportes y art\'iculos.
\item Aprendan, se familiaricen y hagan uso del lenguaje Python para crear peque\~nos programas que puedan interactuar con informaci\'on sobre datos de an\'alisis, reacciones o mol\'eculas para facilitar tareas repetitivas.
\item Se familiaricen con algunos de los algoritmos m\'as modernos en el campo de la computaci\'on cient\'ifica y an\'alisis de datos, y que sepan aplicarlos dependiendo de la situaci\'on que se les presente.
\item Conozcan y se familiaricen con el paquete para c\'alculos num\'ericos SciPy y operaciones qu\'imicas RDKit, para interactuar con diferentes tipos de archivos y extraer o calcular propiedades f\'isicas y qu\'imicas de las mol\'eculas.
\item Conozcan y hagan uso de sistemas de control de revisi\'on Git para llevar registro en la redacci\'on de informes, datos en investigaci\'on y/o avances en desarrollo de software.
\item Conozcan y se familiaricen con los algoritmos para el an\'alisis de secuencias aminoac\'idicas, c\'omo cuantificar su similitud y el uso de acoplamiento molecular en el an\'alisis de afinidad de mol\'eculas peque\~nas a prote\'inas o enzimas.
\item Aprendan a realizar c\'alculos de mec\'anica cu\'antica para la determinaci\'on te\'orica de propiedades fisicoqu\'imicas y posteriormente representar las estructuras electr\'onicas y moleculares.
\item Aprendan a realizar c\'alculos muy simples de din\'amica molecular para la determinaci\'on te\'orica de propiedades moleculares como la energ\'ia y la estructura de mol\'eculas en soluci\'on.
\item Realicen determinaciones de descriptores moleculares mediante flujos de trabajo, y conozcan sobre algoritmos predictivos al trabajar con muchos datos.
\item Analizen y apliquen \emph{scripts} o rutinas que les permita simplificar tareas de bioinform\'atica.
\item Conozcan y hagan uso de bases de datos en l\'inea y aplicaciones web que permiten obtener informaci\'on sobre secuencias de ADN, doblamiento de secuencias aminoac\'idicas por homolog\'ia y determinaci\'on de \'arboles filogen\'eticos.
\item Conozcan y se familiaricen con todo el software de fuente abierta que se puede usar en qu\'imica:
	\begin{enumerate}
	\item Avogadro: Programa para dibujar mol\'eculas en 3D y optimizar las mismas utilizando mec\'anica molecular.
	\item Firefly: Programa para realizar c\'alculos de mec\'anica cu\'antica, optimizaci\'on de geometr\'ia, generaci\'on de espectros y datos termodin\'amicos te\'oricos.
	\item Jupyter: Plataforma para realizar cuadernos electr\'onicos de c\'alculo.
	\item AutoDock Tools: Suite de programas para configurar los par\'ametros de un \emph{docking} en AutoDock o AutoDock Vina.
	\item AutoDock Vina: Programa para realizar simulaciones de acoplamiento intermolecular en la determinaci\'on de energ\'ias de acomplejamiento.
	\item Chimera: Visualizaci\'on de biomol\'eculas, c\'alculo de interacciones, alineado, comparado y b\'usqueda de mol\'eculas similares por medio del m\'etodo BLAST y haciendo uso de otras herramientas para representaci\'on.
	\item VMD: Programa para visualizar mol\'eculas, preparar din\'amicas moleculares para ejecuci\'on en NAMD y visualizar los resultados.
	\item NAMD: Programa para realizar simulaciones de din\'amica molecular, y as\'i  determinar estructuras y energ\'ias de mol\'eculas solvatadas.
	\item RDKit: Herramienta para calcular descriptores moleculares, similitud y relaci\'on estructura-actividad.
	\item KNIME: Plataforma para an\'alisis de datos de manera estad\'istica, inteligente y sencilla.
	\item OpenBabel: Herramienta para interconversi\'on de formatos para representar mol\'eculas, optimizaci\'on de geometr\'ia y c\'alculo de algunas propiedades.
	\item wxMacMolPlt: Programa para visualizar los resultados obtenidos de c\'alculos mec\'anico-cu\'anticos realizados con GAMESS o Firefly.
	\item CHEMBL-EBI: Plataforma web que permite la realizaci\'on de varios c\'alculos bioinform\'aticos.
	\end{enumerate}
\end{enumerate}

\section{Contenido}
\begin{enumerate}
\item \textbf{Sistema Operativo Linux}\\ Entorno gr\'afico, sistema de ficheros, diferencias con otros sistemas, instalaci\'on y manejo de software.
\item \textbf{Shell Unix}\\ Comandos b\'asicos para manejo de archivos, tubos y rutinas sencillas.
\item \textbf{Bases de Datos}\\ Organizaci\'on de una base de datos relacional, \'algebra relacional b\'asica y comandos SQL para interactuar con ellas.
\item \textbf{Lenguaje \LaTeX\ }\\ Comandos b\'asicos en la elaboraci\'on de informes t\'ecnicos, control de estilos, funciones y manejo de paquetes.
\item \textbf{Ambientes de Desarrollo Integrado}\\ Ventajas y desventajas al desarrollar sobre los ambientes Ecplipse, Geany, IDLE y NetBeans y otros editores de texto como Gedit/Pluma, Kate, nano, TexMaker y Sublime Text.
\item \textbf{Lenguaje Python}\\ Comandos b\'asicos, rutinas sencillas y peque\~nos programas que lleven al mejor uso y comprensi\'on de los paquetes para ciencia SciPy y Cinfony, y su implementaci\'on en investigaci\'on y en educaci\'on.
\item \textbf{Sistema de Control de Revisi\'on}\\ Creaci\'on, clonado, descargado, cometido y empujado de proyectos mediante Git y sus formas de visualizado en l\'inea.
\item \textbf{Algoritmos}\\ Desde m\'etodos num\'ericos sencillos (e.g. bisecci\'on, Newton-Raphson) hasta algoritmos de optimizaci\'on (MonteCarlo, Metropolis, Gen\'etico). Su uso y condiciones de uso.
\item \textbf{Paquetes de Software}\\ Uso de estos en la determinaci\'on de propiedades moleculares, optimizaci\'on de geometr\'ias, predicci\'on de conformaciones en acomplejamientos, creaci\'on de modelos y an\'alisis de datos.
\end{enumerate}

\section{Cronograma}
Cada d\'ia se desarrollar\'a el taller durante 4 horas, en dos segmentos de 2 horas para tener oportunidad de poner en pr\'actica todo el contenido y aclarar dudas.

\begin{itemize}
\item \textbf{Sesi\'on 1}: Introducci\'on al taller, entrega e instalaci\'on de software, introducci\'on al ambiente Linux y uso de la consola (l\'inea de comando Bash).
\item \textbf{Sesi\'on 2}: Introducci\'on a las bases de datos, lenguaje SQL y su uso en una base de datos SQLite.
\item \textbf{Sesi\'on 3}: Introducci\'on al uso de \LaTeX\ como herramienta para hacer reportes e informes t\'ecnicos.
\item \textbf{Sesi\'on 4}: Introducci\'on al lenguaje Python, l\'ogica, variables, manejo de errores y desarrollo de programas peque\~nos.
\item \textbf{Sesi\'on 5}: Introducci\'on al lenguaje Python, declaraci\'on de funciones, ciclos y condiciones. Ambientes integrados de desarrollo, ventajas y desventajas.
\item \textbf{Sesi\'on 6}: Paquetes en Python, orientaci\'on a objetos y algoritmos (m\'etodos num\'ericos heur\'isticos y metaheur\'isticos).
\item \textbf{Sesi\'on 7}: Introducci\'on al sistema de control de revisiones Git, manipulaci\'on de proyectos y acceso a los mismos en Internet.
\item \textbf{Sesi\'on 8}: Manipulaci\'on de formatos de archivos, c\'alculo de algunas propiedades, algoritmos de b\'usqueda para secuencias de ADN o de amino\'acidos, BLAST, visualizaci\'on, edici\'on y alineado de biomol\'eculas en UCSF Chimera, propiedades de p\'eptidos y acoplamientos moleculares (docking) utilizando AutoDock Vina.
\item \textbf{Sesi\'on 9}: Dibujo, optimizaci\'on y preparaci\'on de c\'alculos en Avogadro. Determinaci\'on de propiedades termodin\'amicas y orbitales moleculares mediante Firefly. Representaci\'on de los resultados de Firefly mediante wxMacMolPlt.
\item \textbf{Sesi\'on 10}: Obtenci\'on y manipulaci\'on de datos, c\'alculo de descriptores moleculares, construcci\'on de modelos QSAR, uso de algoritmos para \emph{machine learning} y manejo de flujos de trabajo con KNIME.
\item \textbf{Sesi\'on 11}: Uso de archivos FASTA, an\'alisis de \emph{scripts} para el trabajo con secuencias de ADN o de amino\'acidos y uso de herramientas web para la determinaci\'on de estructuras y propiedades de las mismas.
\item \textbf{Sesi\'on 12}: Modelaci\'on de la termodin\'amica de una reacci\'on utilizando c\'alculos mec\'anico-cu\'anticos, determinaci\'on del espectro IR y la energ\'ia HOMO-LUMO del producto.
\item \textbf{Sesi\'on 13}: Proyectos orientados a: bioinform\'atica, dise\~no de medicamentos, dise\~no de materiales y dise\~no \textit{de novo} en general.
\end{itemize}

\section{Materiales y Requisitos}
\begin{itemize}
\item El taller est\'a enfocado para poder ser comprendido sin una base previa en ninguno de los temas a tratar. Sin embargo, est\'a dise\~nado y orientado para personas estudiando qu\'imica y/o alguna de sus ramas.
\item Se requerir\'a que quien atienda al taller provea una memoria USB de m\'inimo 16GB de capacidad para entregarle en ella el software con el que se trabajar\'a.
\item El taller no tendr\'a un costo, pero se requiere un m\'inimo de 10 personas para llevarlo a cabo.
\item Se requeri\'a que quien atienda al taller lleve consigo una computadora laptop (no tablet) para realizar los ejercicios propuestos\footnote{Esto si no se cuenta con infraestructura (hardware) por parte de la instituci\'on en donde se realizar\'a el taller.}.
\item Todo el material se manejar\'a de manera electr\'onica, por lo que antes de cada sesi\'on se le estar\'a enviando a quienes asistan todo lo referente al contenido cubierto esa sesi\'on.
\item Antes de proceder a la siguiente sesi\'on, se llevar\'a a cabo un examen corto para que quien haya atendido al taller pueda determinar qu\'e tanto aprendi\'o y en d\'onde tiene deficiencias a\'un.
\item Al final del taller se har\'a entrega de un diploma de participaci\'on.
\end{itemize}

\bibliographystyle{apacite}
\setlength{\bibindent}{2em}

\nocite{*}

\bibliography{TC3Q}

\end{document}