\documentclass[10pt,letterpaper]{article}
\usepackage[latin1]{inputenc}
\usepackage[spanish]{babel}
\usepackage{amsmath}
\usepackage{amsfonts}
\usepackage{amssymb}
\usepackage{makeidx}
\usepackage{graphicx}
\usepackage{listings}
\usepackage{color}
\usepackage[left=2cm,right=2cm,top=2cm,bottom=2cm]{geometry}
\author{Rony J. Letona}
\title{Taller de Qu\'imica Computacional Aplicada: D\'ia 2}

\definecolor{light-gray}{gray}{0.90}

\newcommand{\tab}[1]{\hspace{.2\textwidth}\rlap{#1}}

\newcommand{\inlinecode}[1]{
\colorbox{light-gray}{\texttt{#1}}
}

\newsavebox{\selvestebox}
\newenvironment{Code}
{
\begin{lrbox}{\selvestebox}%
\begin{minipage}{\dimexpr\columnwidth-2\fboxsep\relax}
\fontfamily{\ttdefault}\selectfont
}
{\end{minipage}\end{lrbox}%
\begin{center}
\colorbox{light-gray}{\usebox{\selvestebox}}
\end{center}
}

\begin{document}
\maketitle

\section{Ejercicios con el Sistema de Control de Revisi\'on - Git}
Al comenzar a trabajar en un proyecto en computaci\'on, generalmente se trabaja con archivos sencillos de texto. Estos tienen el c\'odigo que escribimos y al compilarlos/interpretarlos y ejecutarlos, resultan en programas. Todo funciona as\'i. A veces el c\'odigo en los archivos es comprensible, a veces no lo es porque no es para que lo entendamos nosotros sino el ordenador. Sin embargo, a todos nos ha pasado que borramos un archivo importante o que cambiamos algo que no deb\'iamos de cambiar y no hay manera de recuperarlo. Desde documentos como reportes de laboratorio hasta cartas, siempre nos pasa en alguna ocasi\'on que perdemos informaci\'on importante. Para ello se dise\~n\'o una alternativa.\\

Git es la soluci\'on m\'as eficiente para ello. Es un sistema en el que cada revisi\'on o versi\'on de un archivo se va guardando gradualmente, se lleva registro de cu\'ales cambios se hicieron y de cu\'ando se hicieron. Es el cuaderno de laboratorio digital! No nos salvar\'a una tarea si repentinamente desconectamos el ordenador o se interrumpe la energ\'ia el\'ectrica. Es un sistema para llevar autom\'aticamente registro del trabajo realizado. Lo usan grandes empresas, as\'i como peque\~nas iniciativas en sus proyectos. Y no se tiene que limitar a software, puede tratarse de libros, art\'iculos, bases de datos peque\~nas y cualquier otra cosa que sufra cambios con el tiempo o requiera ser compartida. Por ello, y por otras razones que veremos m\'as adelante, vamos a hacer algunos ejercicios con el sistema git.

\subsection{Configuraci\'on de Git}
Lo primero que debemos hacer con una nueva pieza de software es configurarla. Por ello, vamos a aprender sobre algunas de las cosas importantes que debemos tener para comenzar. Antes de comenzar con esta parte, aseg\'urate de tener una cuenta en GitHub.

\subsection{Creando y Clonando un Repositorio}

\subsection{Estado, Diferencias y Log}

\subsection{Pull, Commit y Push}

\subsection{Ramas, Fork y Merge}

\section{Ejercicios con una Base de Datos Relacional - SQLite}

\subsection{Seleccionando Datos}

\subsubsection{Una Consulta Sencilla}

\subsubsection{\emph{Donde} se cumpla una Condici\'on}

\subsection{Operaciones Matem\'aticas}

\subsubsection{Alterando el Resultado}

\subsubsection{Obteniendo Descriptores}

\subsection{Extendiendo Tablas}

\subsection{ORM}

\subsubsection{SQLite}

\subsubsection{Uno m\'as General: SQLAlchemy}

\section{Glosario de comandos sencillos}
\begin{small}

\begin{itemize}
\item 
\end{itemize}
\end{small}

\end{document}