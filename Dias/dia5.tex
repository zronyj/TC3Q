%
% dia5.tex
% 
% Copyright 2014 Rony J. Letona <rony@zronyj.com>
% 
% This program is free software; you can redistribute it and/or modify
% it under the terms of the GNU General Public License as published by
% the Free Software Foundation; either version 2 of the License, or
% any later version.
% 
% This program is distributed in the hope that it will be useful,
% but WITHOUT ANY WARRANTY; without even the implied warranty of
% MERCHANTABILITY or FITNESS FOR A PARTICULAR PURPOSE.  See the
% GNU General Public License for more details.
% 
% You should have received a copy of the GNU General Public License
% along with this program; if not, write to the Free Software
% Foundation, Inc., 51 Franklin Street, Fifth Floor, Boston,
% MA 02110-1301, USA.
%

\documentclass[10pt,letterpaper]{article}
\usepackage[latin1]{inputenc}
\usepackage[spanish]{babel}
\usepackage{graphicx}
\usepackage{hyperref}
\usepackage{amsmath}
\usepackage{amsfonts}
\usepackage{amssymb}
\usepackage{color}
\usepackage{float}
\usepackage[left=2cm,right=2cm,top=2cm,bottom=2cm]{geometry}
\author{Rony J. Letona}
\title{Taller de Qu\'imica Computacional Aplicada: D\'ia 5}
\definecolor{light-gray}{gray}{0.90}

\newcommand{\tab}[1]{\hspace{.2\textwidth}\rlap{#1}}

\newcommand{\inlinecode}[1]{
\colorbox{light-gray}{\texttt{#1}}
}

\newsavebox{\selvestebox}
\newenvironment{Code}
{
\begin{lrbox}{\selvestebox}%
\begin{minipage}{\dimexpr\columnwidth-2\fboxsep\relax}
\fontfamily{\ttdefault}\selectfont
}
{\end{minipage}\end{lrbox}%
\begin{center}
\colorbox{light-gray}{\usebox{\selvestebox}}
\end{center}
}

\newcommand{\Picture}[1]
{
	\begin{figure}[H]
	\begin{flushleft}
	\includegraphics[width=\columnwidth]{#1}
	\end{flushleft}
	\end{figure}
}

\begin{document}
\maketitle

\section{Ejercicios de Programaci\'on con Python}
Para muchos de nosotros, ver la palabra \emph{programaci\'on} ya es motivo suficiente para decir ``Est\'an locos si creen que voy a hacer esto otra vez''. Probablemente ya hemos intentado darle instrucciones a un ordenador en el pasado (posiblemente en el colegio) y no ha sido nada bonito. Sin embargo, ahora no va a ser como en ese entonces.\\

En el taller hemos ido viendo las cosas lento y paso a paso. No nos van a pedir proyectos ni ejercicios traum\'aticos con los que tendremos pesadillas que ni el psic\'ologo querr\'a escuchar. No, esta vez ser\'a mucho m\'as f\'acil y mucho m\'as interesante. Si el programa no corre o se rompe, tranquilos, no pasa nada. Pero para tener todo claro, comencemos con la pregunta de muchos: Para qu\'e programaci\'on?\\

Mucho del software escrito para fines cient\'ificos resulta ser, generalmente, muy espec\'ifico. Este hace una tarea \textbf{muy} bien, pero no permite hacer algunas otras cosas que nos interesan. Sin embargo, mucho del software permite que se le escriba plugins, extensiones o paquetes con los que podemos extender la funcionalidad del programa original. Adem\'as de esto, el poder programar nuestras propias rutinas o pruebas nos da la capacidad de analizar mejor nuestros datos, calcular propiedades espec\'ificas (o a veces nuevas) y llevar a cabo procesos en el orden que nos interesa. En todas las ciencias, esto resulta ser muy pr\'actico.\\

Otra pregunta que tambi\'en puede surgirnos es: Por qu\'e Python como lenguaje de programaci\'on? Hasta hace una d\'ecada m\'as o menos, el lenguaje utilizado para hacer c\'alculos y rutinas en ciencia era Fortran. Este lenguaje, sin embargo, se quedaba atr\'as en lo que se refiere a versatilidad. Solo serv\'ia para c\'alculos y no permit\'ia m\'as. Luego est\'a Pascal o Delphi. Muchos aprendimos a programar en \'el y hasta la fecha se sigue usando para aprender a programar, pero solo para eso; no se utiliza casi nunca en el mundo real. Visual Basic es otro de aquellos lenguajes que algunos aprendimos. Funciona bien casi solo en ambientes Microsoft Windows, lo cual nos limita bastante. Tambi\'en est\'an las alternativas como C y C++. Estos dos, seg\'un muchos educadores, nos confunden m\'as a los estudiantes con la sintaxis de lo que nos pueden servir para aprender la l\'ogica de programaci\'on. El caso es similar con Java, aunque este tiene una ventaja que los anteriores no: un programa corre sobre cualquier sistema operativo; no hace falta compilarlo para cada caso. JavaScript es el famoso lenguaje utilizado en p\'aginas web y, con el nacimiento de Node.JS, ahora se puede utilizar para programas formales que no necesitan un navegador. El problema es que todav\'ia no cuenta con tantos paquetes para extender su funcionalidad. Finalmente, desde el 7 de julio de 2014, es un hecho que la mayor\'ia de universidades de renmbre en Canad\'a, Estados Unidos y varias en Europa utilizan Python como el lenguaje a aprender en sus cursos b\'asicos de programaci\'on para todas las carreras.\\

Las ventajas que tiene Python sobre los dem\'as lenguajes son todas las antes mencionadas: versatilidad, cantidad de paquetes, corre en cualquier plataforma, es f\'acil de aprender y es bastante universal. Adem\'as, existe bastante documentaci\'on para \'el y mucho del software para QC est\'a parcialmente escrito en \'el. Lo \'unico en lo que tiene una clara desventaja es en su velocidad. Sin embargo, para efectos pr\'acticos, esto no importa tanto ya que no trabajaremos con \emph{demasiadas} operaciones y nuestros procesadores ya son capaces de suplir algunas de las deficiencias en velocidad de este lenguaje. Si dese\'aramos trabajar con un lenguaje m\'as r\'apido y que funcione en todas partes, Java es quiz\'a la siguiente mejor opci\'on, aunque la sintaxis nos exigir\'a ser m\'as rigurosos a la hora de escribir nuestros programas.\\

Para poder responder mejor a todas nuestras dudas, comencemos con los ejercicios y vamos a irnos dando cuenta c\'omo es que con pocos comandos e informaci\'on, logramos hacer desde peque\~nos c\'alculos hasta an\'alisis de datos y predicci\'on de propiedades.\\

\section{Operaciones B\'asicas}
Vamos a comenzar con lo b\'asico. Vamos a abrir nuestra l\'inea de comando, vamos a escribir \inlinecode{python} en ella y vamos a presionar enter. Inmediatamente veremos la interfaz lista para comenzar: \inlinecode{$>>>$} Una cosa que tenemos que tomar en cuenta es que vamos a tener que ir re-aprendiendo a hacer muchas cosas al aprender a programar. Se dice que al aprender a programar, la forma de pensar de una persona cambia un poco. Algo as\'i como cuando aprendemos a ser rigurosos en un an\'alisis.\\

La idea con esta primera parte ser\'a entonces comenzar haciendo ejercicios \textbf{muy} simples. Desde aritm\'etica sencilla hasta almacenar valores en variables. Estas cosas tan sencillas se pueden hacer en calculadoras, pero una cosa que debemos recordar es que un ordenador no es m\'as que eso: una calculadora muy sofisticada. Pero si aprendemos a usarla bien, no habr\'a operaci\'on que no podamos hacer. Comencemos!

\subsection{Aritm\'etica}
Para comenzar vamos a tomar la operaci\'on m\'as b\'asica que aprendimos: sumar. C\'omo se hace esto? Pues es sencillo, hacemos algo as\'i: \inlinecode{4\ +\ 5} y presionamos enter. Como nos damos cuenta, esto es relativamente f\'acil. Es una buena pr\'actica en programaci\'on dejar espacios a los lados del signo de la operaci\'on; en este caso la suma \inlinecode{+}\\

Intentemos ahora realizar una resta. Atrevidamente podremos pensar de una vez en cosas como ``Y qu\'e pasa si la resta tiene un resultado menor a 0?'' Pues la manera de averiguarlo es probando: \inlinecode{8\ -\ 13} El resultado era lo que esper\'abamos? Sigamos probando.\\

Ahora multipliquemos. Esto tampoco se ve dif\'icil: \inlinecode{14\ *\ 86} De hecho, podemos notar que la velocidad de respuesta es igual que la de una calculadora. Si as\'i lo deseamos, podemos hacer m\'as pruebas que las que aqu\'i estamos poniendo. Por ahora, todo va bien.\\

Llegamos a la divisi\'on. Intentemos lo siguiente entonces: \inlinecode{16\ /\ 3} Obtuvimos una respuesta? Perfecto! La respuesta era la que esper\'abamos? No? Por qu\'e no? Por qu\'e ser\'a que la respuesta no sali\'o como esper\'abamos? Despu\'es de todo esto solo fue una simple divisi\'on de enteros. Ah! All\'i est\'a la clave! Fue una divisi\'on de \emph{enteros}. Eso significa que el ordenador todav\'ia no sabe que estamos trabajando con decimales, as\'i que no los toma en cuenta. C\'omo hacer que esto funcione entonces? Pues indic\'andole que debe de considerar que los n\'umeros pueden tener decimales, as\'i: \inlinecode{16.0\ /\ 3}\\

Ya vimos las 4 operaciones b\'asicas que conocemos. Ahora, recordando algo que vimos el d\'ia 2, vamos a ver c\'omo se calculan los residuos de una divisi\'on de enteros mediante la operaci\'on \emph{m\'odulo}: \inlinecode{\%} Si intentamos una divisi\'on de n\'umeros enteros, como lo es \inlinecode{18\ /\ 7}, vamos a obtener \inlinecode{2} como resultado. Pero sabemos que \inlinecode{7\ *\ 2} no es \inlinecode{18} De la divisi\'on anterior obten\'iamos un residuo que podemos calcular as\'i: \inlinecode{18\ \%\ 7} A primera vista esto no se ve de mucha utilidad, pero al trabajar con listas, arreglos o conjuntos de objetos vamos a ver que nos va a servir.\\

Finalmente, vamos a ver una operaci\'on que no es tan b\'asica, pero nos puede servir mucho. Para evitar estar multiplicando muchas veces el mismo n\'umero, en matem\'atica se crearon las potencias. Una operaci\'on que se ve sencilla, como $3^{5}$ o $2^{7}$ nos toman un momento al hacerlas en la mente, pero resulta que nuestro ordenador lo puede hacer en mil\'esimas de segundo. Intentemos esas dos operaciones! Para la primera, ingresamos \inlinecode{3\ **\ 5} y observamos el resultado. Efectivamente obtuvimos lo que dese\'abamos. Ahora intentemos la segunda: \inlinecode{2\ **\ 7} Tambi\'en obtenemos algo que nos satisface. Pero, qu\'e pasa si intentamos calcular raices o potencias con decimales? Intentemos dos ejemplos: \inlinecode{3.14159265359\ **\ (1/2.0)} y \inlinecode{5\ **\ 2.71828182846} Muy interesante! Con esto notamos que podemos hacer muchas m\'as operaciones. Tomemos nota y enumeremos las operaciones que podemos llevar a cabo con todo lo que acabamos de aprender.

\subsection{Variables}
Por ahora vamos viendo que el ordenador se comporta como una plataforma que nos permite hacer operaciones matem\'aticas. Por eso, a la hora de escuchar la palabra \emph{variable}, es natural que pensemos que estas son como las que incluimos en las funciones (e.g. \emph{x} en $ f \left( x \right) = x^{2} - x - 1 $). Sin embargo, en el ambiente de un ordenador esto es un poco diferente. Las variables que nosotros conocemos pueden no tener un valor al inicio, como acabamos de ver. Pero al trabajar en un ambiente digital, este requiere n\'umeros, valores y algo s\'olido con qu\'e trabajar. Por ello, podemos pensar que las variables aqu\'i ser\'an solo un lugar d\'onde almacenar datos, lo cual no es nada lejano a la realidad.\\

Un ordenador posee un dispositivo llamado memoria. En ella se almacenan datos de manera vol\'atil (i.e. se borran al apagar el sistema). Desde que arrancamos nuestro ordenador hasta que lo apagamos, la memoria (RAM) est\'a alojando informaci\'on que permite que trabajemos en el momento. Es entonces aquel lugar en donde est\'an nuestros documentos antes de que guardemos cambios. Por eso si nos quedamos sin energ\'ia el\'ectrica, todo se pierde. Aparte es el espacio en el disco duro, el cual nos da la capacidad de almacenar datos de manera permanente. Pero en esta secci\'on vamos a ver un concepto r\'apido que nos servir\'a para comprender c\'omo funcionan las variables y otras cosas m\'as adelante.\\

Volviendo a las variables, estas son entonces un espacio en memoria. Algo como una peque\~na caja $ \square $ en donde se puede almacenar un dato particular. Podemos sacarlo de la caja? Claro! Tambi\'en podemos colocar algo diferente all\'i despu\'es. Como espacio en memoria, puede almacenar lo que sea. Eso s\'i, no pueden haber dos datos en la misma caja, y por ende, no pueden haber dos datos almacenados en la misma variable.\\

Pasemos, pues, a almacenar algunos datos en variables. La forma de hacer esto es escribendo algo as\'i: \inlinecode{a = 5} Presionamos enter y ... nada pas\'o? De hecho, si no ha salido ning\'un error, todo est\'a bien. Lo que pasa es que no hay nada que mostrar. El dato fue almacenado y ya. Si queremos ver lo que hay dentro de la variable, podemos escribir lo siguiente: \inlinecode{print a} Esto nos mostrar\'a o \emph{imprimir\'a} en pantalla aquello que hay en la variable \inlinecode{a} Debe de llamarse \emph{a} o \emph{b} o \emph{c}? No, puede llamarse como nosotros querramos! Lo \'unico que hay que tomar en cuenta en una variable es que su nombre no puede comenzar con un n\'umero, y esta no puede tener caracteres con tildes, di\'eresis, etc. Tampoco e\~nes. Otra cosa importante es que no podemos ponerle dos nombres a una variable. Eso significa que no puede haber espacios en el nombre de una variable. Con esto en mente, intentemos otra cosa: \inlinecode{mi\_segunda\_variable = 1.61803398875} Como podemos ver, esto tambi\'en se vale. Si despu\'es imprimimos el valor de \inlinecode{mi\_segunda\_variable} veremos que todo funciona bien. Intentemos algo m\'as interesante con esto.\\

Vamos a intentar transformar una temperatura de grados Farenheit a Kelvin. Esto implica transformar primero la temperatura de grados Farenheit a grados Celcius, y luego a Kelvin. Para ello, hay 3 n\'umeros importantes que debemos recordar: una pendiente \inlinecode{m = 5.0/9}, un desplacamiento en el eje \emph{x} \inlinecode{d = 32} y un desplazamiento en el eje \emph{y} \inlinecode{b = 273.15} Ahora veamos, la f\'ormula ir\'ia algo as\'i:

\begin{itemize}
\item De $^o F$ a $^o C$:\hspace*{1cm} $T_{C} = \frac{5}{9} \left( T_{F} - 32 \right) = m \left( T_{F} - d \right)$
\item De $^o C$ a $K$:\hspace*{1cm} $T_{K} = T_{C} + 273.15 = T_{C} + b$
\end{itemize}

Si sustituimos una dentro de la otra, nos queda una f\'ormula algo as\'i: $T_{K} = m \left( T_{F} - d \right) + b$. Ahora, si nos dicen que la temperatura hoy es de $73^o F$ lo que hacemos nosotros, habiendo almacenado las variables anteriores, es esto: \inlinecode{m * (73 - d) + b} El resultado sale de inmediato y nos complace saber que lo hicimos bien.\\

Perfecto! Ahora ya sabemos usar variables. Como \'ultimo ejercicio antes de la siguiente secci\'on, intentemos convertir la siguiente temperatura de Kelvin a grados Farenheit y luego a Celsius: $298.45 K$.

\section{Tipos de Datos}
En qu\'imica, cuando comenzamos a ver de qu\'e est\'a conformada la materia, aprendemos que esta tiene diferentes partes: mol\'eculas, \'atomos, iones, electrones, protones, etc. Aqu\'i sucede algo similar. La informaci\'on en nuestro ordenador se puede descomponer en diferentes partes o \emph{tipos de datos}. Por ahora solo hemos conocido a los n\'umeros enteros y aquellos con punto decimal, pero ahora vamos a comenzar desde un poco m\'as abajo.\\

Ya es cultura general que la informaci\'on en nuestros ordenadores se almacena en forma de unos y ceros, pero esto no es solo cuesti\'on de que si deseamos vamos a ver la informaci\'on codificada as\'i. Lo que tenemos que tomar en cuenta es que estos n\'umeros son c\'odigo binario que codifica a n\'umeros enteros. Estos, a su vez, representan a veces letras y valores num\'ericos o l\'ogicos. Comencemos viendo estos \'ultimos.\\

Al almacenar alg\'un dato en una variable, podemos averiguar qu\'e tipo de dato es el que contiene la variable por medio de un comando sencillo: \inlinecode{type()} Al momento de querer saber de qu\'e tipo de dato se trata, solo aplicamos la funci\'on \textbf{type} a una variable de la manera siguiente: \inlinecode{type(variable)} Conforme vayamos avanzando en este tema, iremos viendo c\'omo es que Python le llama a cada tipo de dato.

\subsection{Booleanos}
Los valores l\'ogicos son solo dos: \emph{verdadero} y \emph{falso}. Si recordamos nuestros cursos de l\'ogica y filosof\'ia, recordamos aquellas tablas de verdad que se nos pon\'ia a hacer para revisar el valor de una proposici\'on. A pesar de que no vamos a hacer tablas de verdad, es bueno recordar c\'omo funcionaban, puesto que aqu\'i vamos a usar muchas de las cosas que all\'i se usaban. Pero bueno, comencemos a ver c\'omo funcionan los valores l\'ogicos.\\

Como ya hab\'iamos mencionado, que exist\'ian solo dos tipos de valor l\'ogico, a los cuales se les denomina \emph{booleanos} por el inventor de estos. Para \textbf{verdadero} vamos a tener \inlinecode{True} y para \textbf{falso} \inlinecode{False}. Estos los podemos colocar en nuestro IDE y vamos a ver que toman un color particular poniendo en evidencia su de que se trata de un valor particular. Pero solos estos no hacen mucho.\\

Algo que siempre se ve de la mano con los valores, son los operadores l\'ogicos. Cuando los conocemos, nos los presentan con nombres un poco complicados como \emph{conjunci\'on}, \emph{disyunci\'on} y \emph{negaci\'on}. En este caso en particular, vamos a verlos en acci\'on de otra forma. Cuando deseamos que dos cosas se cumplan para obtener un resultado verdadero usamos el operador \inlinecode{and}. Cuando deseamos que al menos una de las dos cosas se cumpla para obtener un resultado verdadero usamos \inlinecode{or}. Y cuando deseamos negar un valor en particular, utilizamos \inlinecode{not}. Esto nos va a servir m\'as adelante, cuando querramos tomar decisiones en un programa o una rutina. Para mientras, vamos a ver de d\'onde salen los booleanos.\\

Para tomar una decisi\'on, generalmente revisamos una comparaci\'on de cierto tipo: es esto igual a aquello, es esto m\'as grande que aquello, es esto diferente a aquello, etc. Este tipo de comparaciones se pueden hacer en nuestro ordenador de la siguiente manera:

\begin{enumerate}
\item Si deseamos revisar si dos cosas son iguales: \inlinecode{a == b}
\item Si deseamos revisar si esas dos cosas son diferentes: \inlinecode{a != b}
\item Si deseamos revisar si una cosa es mayor a la otra: \inlinecode{a >\ b}
\item Si deseamos revisar si esa cosa es menor a la otra: \inlinecode{a <\ b}
\item Si deseamos revisar si una cosa es mayor o igual a la otra: \inlinecode{a >= b}
\item Si deseamos revisar si esa cosa es menor o igual a la otra: \inlinecode{a <= b}
\end{enumerate}

Estas operaciones, al igual que las operaciones matem\'aticas, pueden llevarse a cabo combinadas con m\'as operaciones l\'ogicas. Con esto ya podemos escribir cosas como \inlinecode{(a == 0) and (b != 10)} Claro que para hacer todo eso necesitamos que \textbf{a} y \textbf{b} tengan alg\'un valor, pero ahora sabemos que podemos comparar valores de variables y trabajar con ellos. Y antes de pasar a la siguiente secci\'on, conviene decir que Python reconoce este tipo de dato como \inlinecode{bool}

\subsection{N\'umeros}
Este tipo de datos ya lo conocemos ... parcialmente. Sabemos que existen los \textit{enteros} (e.g. 1, -4, 7, 0, 23, ...), los cuales en Python son representados por \inlinecode{int} y los utilizamos generalmente para enumerar cosas. Los enteros llegan hasta donde la memoria de nuestro ordenador nos deje. S\'i, resulta que estos tienen un l\'imite, porque despu\'es de cierto n\'umero, la cifra es tan alta que una cajita de la memoria ya no es suficiente.\\

Despu\'es de esos n\'umeros comienzan los n\'umeros \textbf{largos} (e.g. 9223372036854775808L, -9223372036854775809L, ...). Estos son enteros MUY grandes y si ponemos atenci\'on, tienen una L al final denotando su extra\~no tipo. En Python son representados por \inlinecode{long} y solo nos sirven para c\'alculos con cifras muy grandes. Fuera de eso, cualquier \inlinecode{int} nos sirve para tareas cotidianas.\\

Luego tenemos a un tipo de n\'umeros mucho m\'as conocidos: los de \textit{punto flotante}. Los \emph{qu\'e}? Los n\'umeros de punto flotante. Estos n\'umeros son aquellos que tienen decimales (e.g. 0.333333333333, 3.141592653589, ...). Se les llama de punto flotante porque lo que nuestro  ordenador est\'a haciendo realmente, es guardar el n\'umero sin el punto decimal y guardar la posici\'on del punto en otro espacio en memoria. Al llamar a un n\'umero con decimales, nuestro ordenador va por el n\'umero, y coloca el punto decimal, como flotando, en la posici\'on indicada. Un dato importante a tomar en cuenta es que Python solo trabaja con 12 decimales, y este tipo de dato es representado por \inlinecode{float}\\

Finalmente, tenemos a un tipo de n\'umero con el que uno no se topa muy seguido: los \textit{n\'umeros complejos}\\(e.g. 3 + 4i, -7 - 2i, ...). S\'i, Python puede manejar n\'umeros complejos represent\'andolos como \inlinecode{complex}, solo que en vez de utilizar una \textbf{\^{i}}, se utiliza una \textbf{j} para denotar la parte imaginaria del n\'umero. El uso de los n\'umeros complejos no es algo que se haga muy seguido en el campo de la qu\'imica. De hecho, solo se utilizan casi para c\'alculos cu\'anticos, y a\'un esos los evitamos muchas veces.

\subsection{Cadenas}
Despu\'es de ver que podemos almacenar valores de \emph{verdadero}, \emph{falso} y varios n\'umeros, es solo natural que alguien pregunte: ``Y c\'omo se trabaja con texto?'' El texto es un tipo de dato muy particular. Este es una estructura que consta de muchas peque\~nas partes: las letras. Por ser una estructura alargada y que consta de elementos que la unen, a los fragmentos de texto se les denomin\'o cadenas. En ingl\'es se les conoce como \emph{strings}, por pensar en el mismo principio, pero con un cord\'on.\\

Las cadenas se representan en Python como \inlinecode{str}, y para crearlas, colocamos el texto que deseamos entre comillas o ap\'ostrofes. Intentemos con un ejemplo: Vamos a guardar el texto \textit{Hola mundo!} en una variable. Para ello vamos a ingresar \inlinecode{mi\_variable = 'Hola mundo!'} Solamente! No hay que hacer nada extra\~no o diferente. Ahora, si deseamos mostrar lo que guardamos, solo imprimimos lo que hay en la variable: \inlinecode{print mi\_variable}\\

Perfecto, ya podemos guardar texto. Ahora veamos qu\'e m\'as se puede hacer con \'el, pero intentando entrar ya un poco en qu\'imica. Asumamos, pues, que yo puedo representar a un hidrocarburo como una estructura de Lewis, ignorando sus hidr\'ogenos y sin representar a los enlaces. Si estamos hablando de una cadena de 6 \'atomos de carbono, el resultado ser\'ia as\'i: \inlinecode{hexano = 'CCCCCC'} Como podemos ver, estamos representando a un hidrocarburo con una simple cadena de texto. Qu\'e pasa si deseamos saber cu\'antas letras tiene la cadena? Pues utilizamos una funci\'on para ello: \inlinecode{len(hexano)}\\

Una cosa importante que tambi\'en es conveniente saber es que las cadenas se pueden sumar. Lo que se est\'a haciendo realmente, es concatenarlas. Si de nuestra mol\'ecula anterior queremos hacer un octano, solo nos hacen falta dos carbonos. Entonces, podemos decir que \inlinecode{octano = hexano + 'CC'} Y de esta manera ya tenemos una nueva cadena con nuevos elementos.\\

Intentemos crear otra mol\'ecula: \inlinecode{dietileter = 'CCOCC'} En esta ya incluimos otro \'atomo diferente del carbono. Resulta que deseamos obtener solo ese \'atomo. Para ello tenemos que tomar en cuenta que las cadenas numeran sus caracteres del 0 en adelante. C\'omo as\'i? En este caso, por ejemplo, tendr\'iamos que cada letra se puede representar con el sub\'indice que vemos aqu\'i: $\underset{0}{C} \underset{1}{C} \underset{2}{O} \underset{3}{C} \underset{4}{C}$. Entonces, para referirnos al ox\'igeno en la f\'ormula, lo hacemos de la siguiente manera \inlinecode{dietileter[2]}\\

Otra cosa que podemos hacer es referirnos a segmentos de una mol\'ecula. Si deseamos mostrar todo lo que est\'a despu\'es del ox\'igeno de la mol\'ecula anterior, podemos escribir \inlinecode{dietileter[3:]} O si deseamos lo anterior al ox\'igeno \inlinecode{dietileter[:2]} Notemos ahora que en esto \'ultimo, incluimos al ox\'igeno en nuestra selecci\'on, pero al ejecutar el comando, este no se muestra en el resultado. Esto se debe a que Python no considera al \'ultimo elemento al que nos referimos. Como un \'ultimo ejercicio, vamos a mostrar a un azufre en vez de un ox\'igeno utilizando solo lo que hemos visto hasta ahora: \inlinecode{print dietileter[:2] + 'S' + dietileter[3:]}\\

Antes de proseguir, solo vamos a aclarar algo. Esta forma de representar mol\'eculas a trav\'es de letras es un formato internacional llamado SMILES (\textbf{S}implified \textbf{M}olecular-\textbf{I}nput \textbf{L}ine-\textbf{E}ntry \textbf{S}ystem). Los dobles enlaces se representan con un signo \inlinecode{=}, los triples enlaces con \inlinecode{\#} y para formar ciclos, se numeran los \'atmos que cerrar\'an ese ciclo. Un benceno se ver\'ia entonces de la siguiente manera: \inlinecode{benceno = 'C1=CC=CC=C1'}

\subsection{Listas}
Creacion, acceso, agregado, removido, sustitucion, largo
Datos de un analisis
\subsection{Diccionarios}
Creacion, acceso
Tabla periodica de elementos
\section{Ingresando y Mostrando Datos}
Interactuar con otras cosas: ampliar el mundo a abrir archivos, dar y recibir datos
\subsection{Por Parte del Usuario}
Ingresar numeros, ingresar cadenas, imprimir
Convertir temperaturas otra vez
\subsection{Desde Archivos}
Abrir, cerrar, leer, escribir
Guardar una molecula (SMILES)
\section{Funciones y Paquetes}
Algunas cosas no las podemos hacer solo con esto, vamos a hacerlo muchas veces o es deseable ordenar
\subsection{Funciones Matem\'aticas}
Absoluto, suma, minimo, maximo
Hallar minimo y maximo en archivo con muchos datos
\subsection{Paquetes}
import from X import Y, String, Math, NumPy, SciPy
Hallar pendiente de componentes y punto (vector)
\subsection{Nuestras Funciones}
Definir, devolver y ambito de las variables
Calcular energia libre de Gibbs
\section{Condiciones y Tareas Repetitivas}
Necesidad de eleccion dependiendo de un caso particular y de repetir en base a una condicion
\subsection{Decisiones}
if elif else
Unidad de temperatura en energia libre de Gibbs
\subsection{Repetir mientras llega la Condici\'on}
while break
Revisar molecula hasta hallar halogeno (SMILES)
\subsection{Ciclos}
for in
Calcular media y desviacion estandar de una lista de datos
\section{M\'as All\'a}
Ahorita solo se lleva lo esencial y basico, las posibilidades son mucho mayores
\subsection{Nuestros Paquetes}
Funciones y variables en un archivo = paquete
Convertir temperaturas en un paquete
\subsection{Matem\'atica Avanzada}
Mostrar sympy y numpy
Calcular derivadas, integrales y resolver una matriz cargada en un archivo
\subsection{Qu\'imica en el Ordenador}
Mostrar rdkit
Cargar molecula como SMILES de archivo, colocar Hs, optimizar en 3D, calcular E y guardar datos
\section{Paquete para Qu\'imica sin usar Fortran, Java, C o C++}
Los de ahora no son 100\% Py
\section{Comentarios Finales}
Felicidades
Investigar mas paquetes
Semana entrante: a jugar con quimica y algoritmos

\section*{Licencia}

\noindent \includegraphics{img/cc_big.png}

\noindent Taller de Qu\'imica Computacional Aplicada by \href{http://github.com/zronyj/TQCA}{Rony J. Letona} is licensed under a \href{http://creativecommons.org/licenses/by-sa/4.0/}{Creative Commons Attribution-ShareAlike 4.0 International License}.
Based on a work at \url{http://github.com/swcarpentry/bc}.

\end{document}