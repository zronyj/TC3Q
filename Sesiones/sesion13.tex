%
% sesion13.tex
% 
% Copyright 2017 Rony J. Letona <zronyj@gmail.com>
% 
% This program is free software; you can redistribute it and/or modify
% it under the terms of the GNU General Public License as published by
% the Free Software Foundation; either version 2 of the License, or
% any later version.
% 
% This program is distributed in the hope that it will be useful,
% but WITHOUT ANY WARRANTY; without even the implied warranty of
% MERCHANTABILITY or FITNESS FOR A PARTICULAR PURPOSE.  See the
% GNU General Public License for more details.
% 
% You should have received a copy of the GNU General Public License
% along with this program; if not, write to the Free Software
% Foundation, Inc., 51 Franklin Street, Fifth Floor, Boston,
% MA 02110-1301, USA.
%

\documentclass[10pt,letterpaper]{article}
\usepackage[latin1]{inputenc}
\usepackage[spanish]{babel}
\usepackage{graphicx}
\usepackage{hyperref}
\usepackage{amsmath}
\usepackage{amsfonts}
\usepackage{amssymb}
\usepackage{color}
\usepackage{float}
\usepackage{upquote}
\usepackage[left=2cm,right=2cm,top=2cm,bottom=2cm]{geometry}
\author{Rony J. Letona}
\title{Taller de Computaci\'on Cient\'ifica para Ciencias Qu\'imicas: Sesi\'on 13}
\definecolor{light-gray}{gray}{0.90}

\newcommand{\tab}[1]{\hspace{.2\textwidth}\rlap{#1}}

\newcommand{\inlinecode}[1]{
\colorbox{light-gray}{\texttt{#1}}
}

\newsavebox{\selvestebox}
\newenvironment{Code}
{
\begin{lrbox}{\selvestebox}%
\begin{minipage}{\dimexpr\columnwidth-2\fboxsep\relax}
\fontfamily{\ttdefault}\selectfont
}
{\end{minipage}\end{lrbox}%
\begin{center}
\colorbox{light-gray}{\usebox{\selvestebox}}
\end{center}
}

\newcommand{\Picture}[3]
{
	\begin{figure}[H]
	\begin{center}
	\caption{#3}
	\includegraphics[scale=#2]{#1}
	\end{center}
	\end{figure}
}

\begin{document}
\maketitle

\section{Dise\~no Molecular}
Estamos en la recta final. Llegamos a la parte que m\'as dese\'abamos, pues es ahora que vamos a aprender a hacer lo que hemos venido esperando desde que comenz\'o el taller. Si bien ha sido un recorrido largo, y hemos aprendido muchas cosas, recordemos que solo hemos rascado la superficie de todo lo que es posible hacer con computaci\'on cient\'ifica. Hoy no vamos a seguir instrucciones por mucho tiempo. Lo que intentaremos ser\'a algo m\'as ambicioso. Intentaremos trabajar con una idea que se nos propone, pero sin nada m\'as que nuestro conocimiento adquirido y un peque\~no listado de pasos. Vamos a contar con 3 proyectos, y ser\'ia ideal si logramos hacer los 3.\\

Felicidades por haber llegado tan lejos en el mejor taller de computaci\'on cient\'ifica orientado a qu\'imica del pa\'is! Los contenidos son varios y la intensidad es grande. Esperamos que hayas disfrutado cada sesi\'on y que hayas aprendido mucho sobre lo que se puede y \emph{no} se puede hacer. Sin embargo, lo que m\'as nos agrada ha sido todo lo que tu nos has contribuido en el desarrollo del mismo. Sin ti esto no pudo ser posible.\\

Toma ahora lo que sabes, usa tu creatividad y tu ingenio, y sal a hacer ciencia de una nueva manera que antes no cre\'ias se pod\'ia. Comencemos entonces con la sesi\'on final del TC$^3$Q!

\subsection{Bioinform\'atica}
En la carpeta de datos del taller hallar\'as una carpeta de nombre \textit{Final}. En ella, a su vez, hallar\'as otra de nombre \textit{BI}. De all\'i tomar\'as el archivo FASTA que se te ofrece y te propondr\'as a estudiarlo de la siguiente manera utilizando lo que ya has aprendido.\\

\begin{itemize}
\item Hallar\'as, por medio de la informaci\'on en el FASTA y herramientas vistas anteriormente, qu\'e codifica el FASTA y para qu\'e sirve esa mol\'ecula. Puedes tambi\'en utilizar \href{http://www.uniprot.org/}{UniProt}.
\item Buscar\'as si existen alternativas ya dobladas de la mol\'ecula y, de ser as\'i, las comparar\'as realizando alineaciones con UCSF Chimera.
\item De no hallar alternativas, utilizar\'as SWISS-MODEL para doblar el FASTA y obtener un nuevo modelo de la mol\'ecula en cuesti\'on. Se te recomienda revisar la \textit{sugerencia} en el folder \textit{Data/Final/BI} si buscas un modelo completo.
\item En Chimera, buscar\'as limpiar tu mol\'ecula (ya sea una que hayas hallado doblada o tu nuevo modelo doblado con SWISS-MODEL) y preparar\'as un docking con cada una de las mol\'eculas halladas en \textit{Data/Final/BI/Ejemplos}.
\item Una vez hayas concluido con esto. Procede a guardar toda tu investigaci\'on en un nuevo repositorio propio de GitLab.
\item Reportar\'as tus resultados en un \textbf{peque\~no} documento \LaTeX\ (menos de una p\'agina) como \textbf{Estudio Exploratorio \emph{in silico} del sitio activo de los agonistas del adrenorreceptor $\beta 1$ Humano}.
\end{itemize}

\subsection{Estudio de Medicamentos}
Como nos habremos dado cuenta, acabamos de hacer una peque\~na investigaci\'on! Intentemos continuar, sin quedarnos atr\'as, al momento de dise\~nar un nuevo medicamento. Podemos comenzar abriendo KNIME y entrando a \href{http://www.swisssimilarity.ch/}{Swiss  Similarity}. Lo que haremos ahora ser\'a ensamblar un modelo QSAR.\\

\begin{itemize}
\item Comenzar\'as abriendo alguna de las mol\'eculas de las que hiciste docking anteriormente en Swiss Similarity. All\'i, buscar\'as dentro de la base de datos \textbf{GPCR Ligands (ChEMBL)} y utilizar\'as la columna \textbf{Combined}.
\item Al haber terminado la b\'usqueda, descargar\'as los resultados en formato CSV y los importar\'as a KNIME con el nodo \textit{Read File}.
\item Existe un peque\~no archivo en \textit{Data/Final/DD} llamado \textit{validos.csv}. \'Abrelo con KNIME tambi\'en y utiliza el nodo \textit{Group By} para filtrar los resultados del primer archivo.
\item Haciendo uso de nodos RDKit, agrega hidr\'ogenos, optimiza y calcula descriptores \textbf{Chi} de todas las mol\'eculas de tu archivo. Esto \'ultimo con el nodo \textit{RDKit Descriptor Calculator}.
\item Utilizando nodos Python, construye un modelo de Machine Learning que utilice el modelo \textbf{SVR} de la librer\'ia \textbf{sklearn} a partir de los descriptores que calculaste y de la columna \textit{values}.
\item Observa las mol\'eculas de las que hab\'ias realizado docking utilizando PyMOL. Ahora dibuja, haciendo uso de \href{https://marvinjs-demo.chemaxon.com/latest/demo.html}{MarvinJS}, 5 nuevas mol\'eculas que creas pueden funcionar como medicamentos en el mismo receptor.
\item Guarda tus mol\'eculas en formato SMILES y transf\'ormalas a formato MOL y MOL2 con OpenBabel haciendo uso de la optimizaci\'on en 3D.
\item Importa tus nuevas mol\'eculas MOL en KNIME y, con nodos RDKit, agrega hidr\'ogenos, optimiza y calcula descriptores \textbf{Chi} de las mismas.
\item Utiliza el modelo que hab\'ias dise\~nado antes para predecir su actividad.
\item Realiza docking de tus 5 anteriores mol\'eculas utilizando Chimera.
\item Una vez hayas concluido con esto. Procede a guardar toda tu investigaci\'on en el repositorio de GitLab.
\item Reportar\'as nuevamente tus resultados en un nuevo documento \LaTeX\ (menos de una p\'agina) como \textbf{Dise\~no \emph{in silico} de agonistas del adrenorreceptor $\beta 1$ Humano}.
\end{itemize}

\section*{Licencia}

\noindent \includegraphics{img/cc_big.png}

\noindent Taller de Computaci\'on Cient\'ifica para Ciencias Qu\'imicas by \href{http://github.com/zronyj/TC3Q}{Rony J. Letona} is licensed under a \href{http://creativecommons.org/licenses/by-sa/4.0/}{Creative Commons Attribution-ShareAlike 4.0 International License}.

\end{document}